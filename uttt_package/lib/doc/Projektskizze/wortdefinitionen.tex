\acsetup{first-style=short}
\acsetup{first-style=footnote}


\DeclareAcronym{Jira}{%
  short=Jira,
  short-plural-form=Jiras,
  long=Jira ist Tool zur Fehlerverwaltung eines Projektes.,
  long-plural=
}
\DeclareAcronym{Bamboo}{%
  short=Bamboo,
  long=Bamboo ist ein Tool für Testautomatisierung.
}
\DeclareAcronym{Lizenz17}{%
  short=Lizenz 17,
  long=Lizenz 17 ist die Entwicklerlizenz. Mit dieser hat man alle kaufpflichtigen Optionen aktiviert.
}
\DeclareAcronym{F341}{%
  short=F341,
  long=Programm um in der Finanzbuchhaltung Taxonomien zu importieren
}
\DeclareAcronym{Movietest}{%
  short=Movietest,
  short-plural=s,
  long={Ein Movietest ist ein automatisierter UI-Test. Sprich der Test klickt sich, wie ein Benutzer es tun würde, durch das Programm}
}
\DeclareAcronym{DCZ}{%
  short=DCZ,
  short-plural=s,
  long={DCZ-Dateien sind vorgefertigte UIs, welche eine fixe Struktur haben. Diese kann man einfach importieren und einem Fenster hinzufügen. Sie lassen sich von Hand zusammenklicken.}
}
\DeclareAcronym{AnchoringLayoutPane}{
	short=AnchoringLayoutPane,
	short-plural=s,
	long={Einem AnchoringLayoutPane werden bestimmte Grössen gesetzt. Das Pane passt sich nicht den Child-Elementen an, sondern die Elemente werden an fixe Orte gesetzt und können sich nicht, je nach Platzmenge, neu ausrichten}
}
\DeclareAcronym{UI}{%
  short=UI,
  short-plural=s,
  long={Ein User Interface (deutsch: Benutzeroberfläche) ist eine Schnittstelle, über die eine Person eine Software oder Hardware kontrollieren kann.}
}
\DeclareAcronym{Entry}{
	short=Entry,
	long={Ein Entry besteht aus einem Label (Einfaches Textfeld) und einem Eingabefeld. Dies kann Test, Nummern, Dropdownliste und ähnliches sein.}
}
\DeclareAcronym{ProgressMaker}{
	short=ProgressMaker,
	short-plural=,
	long={Wenn im Code mehrere Datensätze bearbeitet werden, und dies nicht in jedem Fall sofort abgeschlossen ist, muss ein ProgressMaker erstellt werden. Dieser hat einen Ladebalken und zusätzlich noch Loginfos, welche auf Fehler hinweisen.}
}
\DeclareAcronym{API}{
  short=API,
  short-plural=s,
  long={API ist ein Akronym für Application Programming Interface (englisch für Applikationsprogrammschnittstelle). Eine Schnittstelle, die dem Programmierer Funktionen der Hardware, des Betriebssystems, eines Frameworks oder einer Standardbibliothek zugänglich macht.}
}
\DeclareAcronym{trunk}{
  short=trunk,
  short-plural=,
  long={Trunk (deutsch: Stamm) ist die aktuellste Version auf welcher normalerweise entwickelt wird. Neuerungen und Korrekturen werden typischerweise auf trunk entwickelt und anschliessend auf die jeweiligen betreffenden Versionen gemerget.}
}
\DeclareAcronym{merg}{
  short=merg,
  short-plural=,
  long={Merge ist der Vorgang des Abgleichens mehrerer Änderungen, die an verschiedenen Versionen derselben Datei getätigt wurden.}
}
\DeclareAcronym{DesignCockpit}{
  short=DesignCockpit,
  short-plural=,
  long={Das DesignCockpit ist das Programm für die Erstellung von DCZs.}
}
\DeclareAcronym{Cloudtheme}{
  short=Cloudtheme,
  short-plural=,
  long={Das ist der Name des neuen Looks, welche die Abacusprogramme seit der Version 18 bekommen haben.}
}
\DeclareAcronym{Vorgabe} {
	short=Vorgabe,
	long={In Vorgaben werden die aktuellen Einstellungen gespeichert. Dies wird vor allem bei Reportprogrammen genutzt. So lassen sich Auswertungseinstellungen Gruppenweise speichern.}
}
\DeclareAcronym{CatalogPage} {
	short=CatalogPage,
	long={Bei vielen Abacusprogrammen wird links am Rand ein Katalog aufgebaut. In diesem lassen sich verschiedene Panels selektieren, ähnlich wie bei einem UATabPanel.}
}
\DeclareAcronym{Listener} {
	short=Listener,
	long={Listener ist ein Bestandteil des Listener-Pattern. Weitere Infos unter \href{https://de.wikipedia.org/wiki/Beobachter_(Entwurfsmuster)}{Wikipedia}.}
}
\DeclareAcronym{BMS} {
	short=BMS,
	long={Berufsmaturitätsschule}
}
\DeclareAcronym{DIF-Tool} {
	short=DIF-Tool,
	long={Hebt die Unterschiede zwischen zwei Dateien hervor. Somit lassen sich diese Dateien einfach vergleichen.}
}
\DeclareAcronym{IDEA} {
	short=IntelliJ IDEA ,
	long={Die Programmierumgebung, mit welcher in der Abacus entwickelt wird. Siehe \href{https://www.jetbrains.com/idea/}{hier}.}
}
\DeclareAcronym{F22} {
	short=F22,
	long={Programm für Kontoauszüge.}
}
\DeclareAcronym{Epic} {
	short=Epic,
	long={Ein Epic ist eine Gruppe von Jiras, welche alle mit dem gleichen Thema zu tun haben.}
}
\DeclareAcronym{PSQL} {
	short=PSQL,
	long={Pervasive PSQL. PSQL ist eine filebasierte Datenbank ohne! SQL-Unterstützung.}
}
\DeclareAcronym{xmlns} {
	short=XMLNS,
	long={XMLNS steht für XML namespace. Mehr Informationen unter \href{https://de.wikipedia.org/wiki/Namensraum_(XML)}{Wikipedia}}
}
\DeclareAcronym{ESTV} {
	short=ESTV,
	long={Eidgenössische Steuerverwaltung}
}
\DeclareAcronym{ServerGate} {
	short=ServerGate,
	long={Über das Servergate lassen sich Funktionen und Klassen mit anderen Applikationen teilen.}
}
\DeclareAcronym{ELSTER} {
	short=ELSTER,
	long={Apronym für Elektronische Steuererklärung, ist eine Software der deutschen Steuerverwaltung für das elektronische Einreichen der Steuer}
}
\DeclareAcronym{ClassCastException} {
	short=ClassCastException,
	long={Ein Fehler der von Java geworfen wird, wenn man ein Objekt zu einer Subklasse ändern will, von der das Objekt nicht instanziiert.}
}
\DeclareAcronym{IDE} {
	short=IDE,
	long={Integrierte Entwicklungsumgebung, von englisch \textit{integrated development environment}}
}