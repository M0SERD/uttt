\section{Titel des Projektes}
Is it possible to make an algorithm that beats the human brain in Ultimate Tic Tac Toe?

\section{Beteiligte}
\begin{tabularx}{\textwidth}{l|X}
\textbf{Person} & \textbf{Funktion} \\\hline
Stefan Jampen & Verfasser des Projektes \\
Damian Moser & Verfasser des Projektes \\
Alejandro Ojeda Gonzalez & Betreuende Lehrperson im Bereich Mathematik \\
Barbara Becher & Betreuende Lehrperson im Bereich Englisch \\
\end{tabularx}

\section{Ausgangslage}
Zurzeit sind wir zwei Lernende, die sich im Bereich Algorithmenprogrammierung noch nicht so gut auskennen. Mit dieser Arbeit wollen wir das ändern, um so unser Wissen für unsere Berufe (Informatik und Elektronik) zu verbessern. Dabei wollen wir die Fächer Mathematik mit Datenanalyse und Englisch mit einer englischen Dokumentation abdecken.

\section{Ziel}
\todo{Hinzufügen}

\subsection{Fragestellung}
\todo{Hinzufügen}

\subsection{These}
Hyptothesen:
Wir denken, dass wir es schaffen einen Algorithmus zu erstellen, der gegen einen menschlichen Spieler gewinnt, weil wir Kenntnisse im Bereich Programmieren haben und motiviert hinter dem Projekt stehen.
Wir denken, dass es möglich ist, dass unser Programm besser sein wird als ein Mensch, weil wir die verschiedenen Spielstände genau vorhersehen können.

\section{Methodische Eigenleistungen}
\todo{Hinzufügen}

\section{Machbarkeitsüberprüfung}
\todo{Hinzufügen}

\section{Arbeitsplan}
\todo{Hinzufügen}


%1. (Arbeits-)Titel des Projekts
%2. Namen der Beteiligten: Verfasserinnen: Namen, Vornamen / Betreuende Lehrpersonen:
%Namen, Vornamen
%1 Wenn im Folgenden vom Coach die Rede ist, ist damit immer auch die Pluralform (2 Coaches) mitgemeint
%Berufsmaturitätsschule St. Gallen
%Leitfaden IDPA Seite 4
%3. Beschreibung der Ausgangslage und des fachlichen Kontexts, auf den die Arbeit Bezug
%nehmen soll
%4. Formulierung der Ziele, der Fragestelllungen und/oder der Thesen bzw. Hypothesen
%5. Formulierung der methodischen Eigenleistung (wie Interviews, Umfragen, Experimente,
%etc.) und Begründung der Methodenwahl bezüglich der Ziele, Fragestellung
%und/oder Thesen bzw. Hypothesen
%6. Abklärung in Sachen Machbarkeit (Verfügbarkeit von Interviewpartner, Institutionen
%wie z.B. Labore, verfügbare Literatur und Quellen)
%7. Ausweis über das beabsichtigte Vorgehen in Form eines Arbeitsplans mit definierten
%Meilensteinen sowie zwei festgelegten Besprechungstermine mit mindestens einem
%Coach.