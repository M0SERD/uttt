\section{Title of the project}
Target oriented application of a SLAP algorithm for Ultimate Tic Tac Toe.

\section{Parties}
\begin{tabularx}{\textwidth}{l|X}
\textbf{Person} & \textbf{Function} \\\hline
Stefan Jampen & Author of the project \\
Damian Moser & Author of the project \\
Alejandro Ojeda González & Supervising teacher in the field of mathematics \\
Barbara Becher & Supervising teacher in the field of English \\
\end{tabularx}

\section{Starting position}
Currently we are two apprentices. Damian is beeing educated as a computer scientist and Stefan as an electronics engineer. We do have programming experience, but not much in algorithm programming. With this work we want to change that and improve our knowledge for our professions. We want to cover the subjects mathematics with programming and data analysis and English with an English documentation.

% Zurzeit sind wir zwei Lernende, die im Bereich Algorithmenprogrammierung noch nicht so viel Erfahrung haben. Mit dieser Arbeit wollen wir das ändern, um so unser Wissen für unsere Berufe (Informatik und Elektronik) zu verbessern. Dabei wollen wir die Fächer Mathematik mit Datenanalyse und Englisch mit einer englischen Dokumentation abdecken.

\section{Goal}
Our main goal is to develop a self-learning algorithm that can play Ultimate Tic Tac Toe against humans. The implementation is realized by a target-oriented application of the alpha-beta pruning algorithm, which is extended by an evolutionary algorithm and self-learning. We also want to analyze the learning process of the algorithm in more detail and check whether there is really an improvement. For simplification, we call the \textbf{s}elf-\textbf{l}earning \textbf{a}lpha-beta \textbf{p}runing algorithm SLAP algorithm.
% (self-learning alpha-beta pruning) 
% Unser Ziel ist es, einen selbstlernenden Algorithmus zu entwickeln, welcher gegen Menschen Ultimate Tic Tac Toe spielen kann. Die Umsetzung realisieren wir durch eine zielorientierte Anwendung des Alpha-Beta-Suche Algorithmus, welcher durch einen Evolutionären Algorithmus erweitert und selbstlernend wird. Der Lernprozess des Algorithmus wollen wir genauer analysieren und überprüfen, ob wirklich eine Verbesserung stattfindet.
%\subsection{Question} % Fragestellung
%\begin{itemize}
%	\item Welche Gewinnquote kann mit unserem Algorithmus erzielt werden?
%	\item Was für Verbesserungen können vorgenommen werden um die 							Gewinnquote zu erhöhen?
%	\item Wie wirkt sich die Tiefe der Suche des Algorithmus auf die 							 		Gewinnquote aus?
%	\item Schneidet unser Algorithmus besser ab als ein menschlicher Spieler?
%	\item Gibt es eine Korrelation zwischen Alter des Spielers und der Gewinnquote des Algorithmus?
%\end{itemize}
%
%\todo{translate}
%
%\subsection{These}
%Hyptothesen:
%Unser Alpha Beta Pruning Algorithmus gewinnt gegen einen menschlichen Spieler.
%
%Wir denken, dass wir es  schaffen, einen Algorithmus zu erstellen, der gegen einen menschlichen Spieler gewinnt, weil wir Kenntnisse im Bereich Programmieren haben und motiviert hinter dem Projekt stehen.
%Wir denken, dass es möglich ist, dass unser Programm besser sein wird als ein Mensch, weil wir die verschiedenen Spielstände genau vorhersehen können.


\section{Own methodological services}
\begin{itemize}
    \item Development of the whole application including the alpha-beta pruning algorithm and the self-learning algorithm.
    \begin{itemize}
	    \item Development of a alpha-beta search algorithm in dart.
	    \item Development of a self-learning algorithm with the concrete organism objects which play against each other in dart.
	    \item Development of a website where you can play Ultimate TicTacToe against our SLAP algorithm
    \end{itemize}
    \item Documentation of the whole application in english.
    \item Evaluation of the results of the learning process of our SLAP algorithm.
    \item Documentation of our evaluation in english.
\end{itemize}


\section{Feasibility check}
\begin{itemize}
    \item research about the alpha-beta pruning algorithm in advance
    \item research about the self-learning algorithm in advance
    \item detailed clarification in terms of feasibility with Alejandro Gonzalez
\end{itemize}

\section{Work schedule}
\begin{enumerate}
    \item complete project sketch and show it to our Coaches
    \item refine our project sketch
    \item set up of our work space (Version Control and GitHub), already done
    \item meet each other to get started with coding, 2. November
    \item document our work
    \item make the website
\end{enumerate}

%1. (Arbeits-)Titel des Projekts
%2. Namen der Beteiligten: Verfasserinnen: Namen, Vornamen / Betreuende Lehrpersonen:
%Namen, Vornamen
%1 Wenn im Folgenden vom Coach die Rede ist, ist damit immer auch die Pluralform (2 Coaches) mitgemeint
%Berufsmaturitätsschule St. Gallen
%Leitfaden IDPA Seite 4
%3. Beschreibung der Ausgangslage und des fachlichen Kontexts, auf den die Arbeit Bezug
%nehmen soll
%4. Formulierung der Ziele, der Fragestelllungen und/oder der Thesen bzw. Hypothesen
%5. Formulierung der methodischen Eigenleistung (wie Interviews, Umfragen, Experimente,
%etc.) und Begründung der Methodenwahl bezüglich der Ziele, Fragestellung
%und/oder Thesen bzw. Hypothesen
%6. Abklärung in Sachen Machbarkeit (Verfügbarkeit von Interviewpartner, Institutionen
%wie z.B. Labore, verfügbare Literatur und Quellen)
%7. Ausweis über das beabsichtigte Vorgehen in Form eines Arbeitsplans mit definierten
%Meilensteinen sowie zwei festgelegten Besprechungstermine mit mindestens einem
%Coach.