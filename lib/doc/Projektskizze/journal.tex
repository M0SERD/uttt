\section{Title of the project}
Target oriented application of a self-learning alpha-beta pruning algorithm for Ultimate Tic Tac Toe.

\section{Parties}
\begin{tabularx}{\textwidth}{l|X}
\textbf{Person} & \textbf{Funktion} \\\hline
Stefan Jampen & Author of the project \\
Damian Moser & Author of the project \\
Alejandro Ojeda Gonzalez & Supervising teacher in the field of mathematics \\
Barbara Becher & Supervising teacher in the field of English \\
\end{tabularx}

\section{Starting position}
Currently we are two apprentices who do not have so much experience in programming algorithms. With this work we want to change that and improve our knowledge for our professions (computer science and electronics). We want to cover the subjects mathematics with data analysis and English with an English documentation.

% Zurzeit sind wir zwei Lernende, die im Bereich Algorithmenprogrammierung noch nicht so viel Erfahrung haben. Mit dieser Arbeit wollen wir das ändern, um so unser Wissen für unsere Berufe (Informatik und Elektronik) zu verbessern. Dabei wollen wir die Fächer Mathematik mit Datenanalyse und Englisch mit einer englischen Dokumentation abdecken.

\section{Goal}
Our goal is to develop a self-learning algorithm that can play Ultimate Tic Tac Toe against humans. The implementation is realized by a goal-oriented application of the alpha-beta pruning algorithm, which is extended by an evolutionary algorithm and self-learning. We want to analyze the learning process of the algorithm in more detail and check whether there is really an improvement.

% Unser Ziel ist es, einen selbstlernenden Algorithmus zu entwickeln, welcher gegen Menschen Ultimate Tic Tac Toe spielen kann. Die Umsetzung realisieren wir durch eine zielorientierte Anwendung des Alpha-Beta-Suche Algorithmus, welcher durch einen Evolutionären Algorithmus erweitert und selbstlernend wird. Der Lernprozess des Algorithmus wollen wir genauer analysieren und überprüfen, ob wirklich eine Verbesserung stattfindet.
\subsection{Question} % Fragestellung
\todo{Hinzufügen}

\subsection{Theses}
\todo{Hinzufügen}

\section{Own methodological services}
\todo{Hinzufügen}

\section{Feasibility check}
\todo{Hinzufügen}

\section{Work schedule}
\todo{Hinzufügen}


%1. (Arbeits-)Titel des Projekts
%2. Namen der Beteiligten: Verfasserinnen: Namen, Vornamen / Betreuende Lehrpersonen:
%Namen, Vornamen
%1 Wenn im Folgenden vom Coach die Rede ist, ist damit immer auch die Pluralform (2 Coaches) mitgemeint
%Berufsmaturitätsschule St. Gallen
%Leitfaden IDPA Seite 4
%3. Beschreibung der Ausgangslage und des fachlichen Kontexts, auf den die Arbeit Bezug
%nehmen soll
%4. Formulierung der Ziele, der Fragestelllungen und/oder der Thesen bzw. Hypothesen
%5. Formulierung der methodischen Eigenleistung (wie Interviews, Umfragen, Experimente,
%etc.) und Begründung der Methodenwahl bezüglich der Ziele, Fragestellung
%und/oder Thesen bzw. Hypothesen
%6. Abklärung in Sachen Machbarkeit (Verfügbarkeit von Interviewpartner, Institutionen
%wie z.B. Labore, verfügbare Literatur und Quellen)
%7. Ausweis über das beabsichtigte Vorgehen in Form eines Arbeitsplans mit definierten
%Meilensteinen sowie zwei festgelegten Besprechungstermine mit mindestens einem
%Coach.