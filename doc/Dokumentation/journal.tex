\section{Introduction}
%introduction of the topic
Algorithms are everywhere. They are a part of our everyday lives without even realizing it. They can be found in computers, smartphones and even in cookery books. But what exactly is an algorithm? According to ... algorithms are generally speaken ... (citation).
In our apprenticeship, we both have a lot to do with algorithms. Damian works as a software engineer, where he writes most of the time software. Stefan is an electronics engineer and his specialisation in the fourth year of education is also software. According to that, we both have a pretty good idea of what algorithms are and what they do. We found out, that each of us is fascinated by the concepts and possibilities algorithmic structures provide. So we came up with the idea, to implement an algorithm by ourselves. 

%goals
We set the goal, to program an algorithm, that can play the game Ultimate TicTacToe. UTTT is a more complex and strategic version of the ordinary TicTacToe. 
Our approach to solve that problem is a so called Self Learning Alpha Beta Pruning Algorithm, also known as SLAP Algorithm. It's a combination of the already existing Alpha Beta Pruning algorithm combined with an own made up extension which is, as the name suggests, self learning. We want to analyse the learning process of the algorithm in more detail and evaluate whether there is really an improvement. This will be the mathematical part of our work.
To experience our algorithm in action, we also wanted to create a website where everybody can play the game against our SLAP Algorithm.
Another intention of our project is to bring the seemingly complex subject in an easy understandable form to the interested reader. We aim to resume our approach of the solution in a comprehensible way. We want to give the reader an insight and a deeper understanding of algorithms, how they work and how they are implemented. With the website, we hope to create an interesting and interactive extension to this paper.
In order to meet the requirements to cover two subjects, we decided to write our paper and the website in English.

%overview methodological approach
The first part of our IDPA will be to implement the game itself and the corresponding SLAP algorithm. 
That includes to find a way to value the states of the game in the most efficient way. This will be the task of the evolutionary algorithm. Only if that part of the SLAP algorithm works fine, the Alpha Beta Pruning is able to work in our favour.
We decided to realise all of that with a modern programming language called Dart from Google. Dart is ...(something about Dart). With Dart we have the flexibility, to write back end and front end source code in the same language. Once everything is set up, we will extend our project, that we are able to track the development of the self learning part. We also plan to build a function into the website, where anybody can train their own organisms.
%overview structure of the paper



